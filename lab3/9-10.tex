\section{作业九}
    \subsection{}
        首先根据文档,要判断缺页是发生在用户模式还是内核模式下,需检查tf_cs的低位,在kern/trap.c的函数page_fault_handler中添加代码如下:

        \begin{lstlisting}[style=CPP]
if ((tf->tf_cs&3) == 0)
    panic("Kernel page fault!");
        \end{lstlisting}[style=CPP]

    \subsection{}
        在kern/pmap.c实现函数user_mem_check,检测线性地址是否有效:

        \begin{lstlisting}[style=CPP]
int
user_mem_check(struct Env *env, const void *va, size_t len, int perm)
{
    // LAB 3: Your code here.
    cprintf("user_mem_check va: %x, len: %x\n", va, len);
    uint32_t begin = (uint32_t) ROUNDDOWN(va, PGSIZE); 
    uint32_t end = (uint32_t) ROUNDUP(va+len, PGSIZE);
    uint32_t i;
    for (i = (uint32_t)begin; i < end; i += PGSIZE) {
        pte_t *pte = pgdir_walk(env->env_pgdir, (void*)i, 0);
        if ((i >= ULIM) || !pte || !(*pte & PTE_P) || ((*pte & perm) != perm)) {
            user_mem_check_addr = (i<(uint32_t)va?(uint32_t)va:i);
            return -E_FAULT;
        }
    }
    cprintf("user_mem_check success va: %x, len: %x\n", va, len);
    return 0;
}
        \end{lstlisting}

        根据注释,用户程序可获得一个线性地址仅当:1)该地址低于ULIM;2)页表给与其权限perm | PTE_P。

    \subsection{}
        接着,在kern/syscall.c的sys_cputs中检查用户空间地址:

        \begin{lstlisting}[style=CPP]
static void
sys_cputs(const char *s, size_t len)
{
    // Check that the user has permission to read memory [s, s+len).
    // Destroy the environment if not.

    // LAB 3: Your code here.
    user_mem_assert(curenv, s, len, PTE_U);

    // Print the string supplied by the user.
    cprintf("%.*s", len, s);
}
        \begin{lstlisting}
        
        此时,执行make run-buggyhello,进程会被销毁,内核不会恐慌。

    \subsection{}
        最后,修改在kern/kdebug.c的debuginfo_eip,对usd,stabs,stabstr分别调用user_mem_check。

        \begin{lstlisting}[style=CPP]
// Make sure this memory is valid.
// Return -1 if it is not.  Hint: Call user_mem_check.
// LAB 3: Your code here.
if (user_mem_check(curenv, usd, sizeof(struct UserStabData), PTE_U))
    return -1;

    ...
// Make sure the STABS and string table memory is valid.
// LAB 3: Your code here.
if (user_mem_check(curenv, stabs, sizeof(struct Stab), PTE_U))
    return -1;
if (user_mem_check(curenv, stabstr, stabstr_end-stabstr, PTE_U))
    return -1;
        \end{lstlisting}

\section{作业十}
    根据文档,上面实现的机制对恶意的用户程序也同样有效,进程会被销毁,内核不会恐慌。
